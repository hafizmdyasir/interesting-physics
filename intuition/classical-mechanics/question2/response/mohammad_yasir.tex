\textbf{Short answer}: Action and reaction pairs act on different bodies and thus, do not cancel each other out.

\textbf{Explanation}: For two forces to cancel each other out, they must act on the same body and their resultant must be zero. Interestingly, this answer is hidden in the statement of the question itself. As per Newton's third law, the action of the boy tossing up the ball causes a reaction, said reaction being the ball exerting a force on \textit{the boy}. Hence, \textbf{action and reaction pairs act on separate bodies, preventing them from cancelling each other out}.

Action and reaction pairs are immensely interesting and have widescale applications. For instance, we apply a force on Earth when we walk, which causes the Earth to push us back. Since the Earth has a mass infinitely larger than our mass, it does not move because of the force we exert but we get to walk forward as a consequence of the reaction we experience.
