This problem was tackled in a research paper titled "The density of a body and special relativity" by Shukri Klinaku, a copy of which, may be seen at \href{https://www.researchgate.net/publication/283213614_The_Density_of_a_Body_and_Special_Relativity}{here}. I will attempt a similar approach here. The short answer is that density is invariant under Lorentz transforms and we will prove the same here.

Let us resolve this apparent paradox without even worrying about the bullet and the fluid. Consider two reference frames $S$ and $S'$, where $S'$ is moving at a relativistic velocity with respect to  S (along the x-axis only). We then consider a cuboidal box , say $B$, with dimensions $x,y,z$ along the x, y, and z-axes respectively in the S frame.

\textbf{For the S-frame}, the density is then given as 
\begin{equation}
    \rho = \frac{m}{V}
    \label{density}
\end{equation}


\textbf{For the S'-frame}, the real catch enters the picture. This time, since the frame is moving at a relativistic pace, not only the volume, but also the mass of the cuboid will get altered. Hence, the density in this frame is $$\rho = \frac{m'}{V'}$$
Where $m'$ is given as 
\begin{equation}
    m' = \frac{m}{\sqrt{1-\frac{v^2}{c^2}}}
    \label{relativisticMass}
\end{equation}
The volume is modified due to length contraction. The length in x direction is $$x' = x \sqrt{1-\frac{v^2}{c^2}}$$
and thus, the modified volume is $$V' = x \sqrt{1-\frac{v^2}{c^2}} \times y \times z$$
\begin{equation}
    V' = V \sqrt{1-\frac{v^2}{c^2}}
    \label{relativisticVolume}
\end{equation}


Let us replace the values of $m$ and $V$ in equation~\ref{density} from equations~\ref{relativisticMass} and~\ref{relativisticVolume}. We get
\begin{align*}
    \rho &= \frac{m}{v}\\
        &= \frac{\frac{m'}{\sqrt{1-\frac{v^2}{c^2}}}}{\frac{V'}{\sqrt{1-\frac{v^2}{c^2}}}}\\
        &= \frac{m'}{V'}\\
    \rho &= \rho '
\end{align*}
Thus, we can see that the length contraction effect is counterbalanced by the mass dilation effect, effectively making the density of any object an invariant quantity regardless of whether the motion is relativistic or not.
