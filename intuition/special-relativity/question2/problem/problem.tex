\subsection{Does the string break?}
	``\ldots Three small spaceships, A, B, and C, drift freely in a region of space remote from other 
	matter, without rotation and without relative motion, with B and C equidistant from A.
	
	On reception of a signal from A the motors of B and C are ignited and they accelerate gently.

	Let ships B and C be identical, and have identical acceleration programmes. Then (as reckoned
	by an observer in A) they will have at every moment the same velocity, and so remain 
	displaced one from the other by a fixed distance. Suppose that a fragile thread is tied 
	initially between projections from B and C. If it is just long enough to span the required
	distance initially, then as the rockets speed up, it will become too short, because of its 
	need to [Lorentz-]Fitzgerald contract, and must finally break. It must break when , at a 
	sufficiently high velocity, the artificial prevention of the natural contraction imposes 
	intolerable stress. 

	Is it really so? \ldots	''\\
	\hspace*{15pt}{\scriptsize Credit: Bell, J.S.,`How to Teach Special Relativity' \textit{Prog. Sci. Culter} \textbf{1} (1976)}